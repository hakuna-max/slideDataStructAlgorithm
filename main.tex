\documentclass{sintefbeamer}

% meta-data
\title{数据结构与算法\\ Data Structure and Algorithm}
\subtitle{Introduction}
\author{\href{mailto:zlp@upc.edu.cn}{ZHANG Luping}}
\date{\today}
\titlebackground{images/background}

% document body
\begin{document}

\maketitle

\begin{frame}

  This template is a secondary creation of \hrefcol{https://www.overleaf.com/latex/templates/sintef-presentation/jhbhdffczpnx}{SINTEF Presentation} template from \hrefcol{mailto:federico.zenith@sintef.no}{Federico Zenith} \vspace{\baselineskip}

  Following a brief introduction written by \hrefcol{mailto:federico.zenith@sintef.no}{Federico Zenith} about how to use \LaTeX\ and beamer to prepare slides. All rights reserved by him\vspace{\baselineskip}

  This template is released under \hrefcol{https://creativecommons.org/licenses/by-nc/4.0/legalcode}{Creative Commons CC BY 4.0} license

\end{frame}

\section{Desired Outcomes}

\begin{frame}
  \frametitle{}

  \begin{itemize}
    \item You have knowledge of the most common abstractions for data collections (e.g., stacks, queues, lists, trees, maps).
    \item You understand algorithm strategies for producing efficient realizations of common data structures.
    \item You can analyze algorithmic performance, both theoretically and experimentally, and recognize common trade-offs between competing strategies.
    \item You can wisely use existing data structures and algorithms found in modern programming language libraries.
    \item You have experience working with concrete implementations for most foundational data structures and algorithms.
    \item You can apply data structures and algorithm to solve complex problems.
  \end{itemize}

\end{frame}

\section{Algorithm}

\begin{frame}{Declarative knowledge vs. Imperative knowledge}

  All knowledge can be thought of as either \textcolor{red}{\textbf{declarative or imperative}}.

  \begin{itemize}[<+->]
    \item \textcolor{red}{\textbf{Declarative knowledge}} is composed of statements of fact.
          \begin{itemize}
            \item \textit{The square root of $x$ is a number $y$ such that $y \times y = x$},
            \item \textit{It is possible to travel by train from Qingdao to Beijing}
          \end{itemize}
    \item \textcolor{red}{\textbf{Imperative knowledge}} is ``how to'' knowledge, or recipes for deducing information.
  \end{itemize}

\end{frame}

\begin{frame}
  \frametitle{A way to compute the square root of a number}

  \pause

  \begin{enumerate}
    \item Start with a guess, $g$.
    \item If $g \times g$ is close enough to $x$, stop and say that $g$ is the answer.
    \item Otherwise, create a new guess by averaging $g$ and $\frac{x}{g}$, i.e., $\frac{(g + \frac{x}{g})}{2}$.
    \item Using this new guess, which we again call $g$, repeat the process until $g \times g$ is close enough to $x$.
  \end{enumerate}

\end{frame}

\begin{frame}
  \frametitle{Definition of \textcolor{red}{algorithm}}

  \begin{itemize}[<+->]
    \item Note that the description of the method is \textbf{a sequence of simple steps, together with a flow of control} that specifies when to execute each step. Such a description is called an \textcolor{red}{\textbf{algorithm}}.\\[5pt]
    \item More formally, \textit{an algorithm is \textcolor{blue}{a finite list of instructions} describing a set of \textbf{computations} that when executed on \textcolor{blue}{a set of inputs} will proceed through a sequence of well-defined states and eventually produce \textcolor{blue}{an output}}.
  \end{itemize}

\end{frame}

\begin{frame}[fragile]{Examples: Perfect cube root}
  This code prints the integer cube root, if it exists, of an integer. If the input is not a perfect cube, it prints a message to that effect.

  \begin{block}{}
    \begin{lstlisting}[language=Python]
      # Find the cube root of a perfect cube
      x = int(input("Enter an integer: "))
      ans = 0
      while ans**3 < abs(x):
          ans += 1
      if ans**3 != abs(x):
          print(x, "is not a perfect cube")
      else:
          if x < 0:
              ans = -ans
          print("Cube root of”, x, "is", ans)
    \end{lstlisting}
  \end{block}
\end{frame}

\begin{frame}[fragile]{Examples: Prime number}
  This code tests whether an integer is a prime number and returning the smallest divisor if it is not.

  \begin{block}{}
    \begin{lstlisting}[language=Python]
      # Test if an int > 2 is prime. If not, print smallest divisor
      x = int(input("Enter an integer greater than 2: "))
      smallest_divisor = None
      for guess in range(2, x):
          if x%guess == 0:
              smallest_divisor = guess
              break
      if smallest_divisor != None:
          print("Smallest divisor of", x, "is", smallest_divisor)
      else:
          print(x, "is a prime number")
    \end{lstlisting}
  \end{block}
\end{frame}

\begin{frame}[fragile]
  \frametitle{Examples: Approximation to the square root of $x$}

  \begin{block}{}
    \begin{lstlisting}[language=Python]
      epsilon = 0.01
      step = epsilon**2
      num_guesses = 0
      while abs(ans**2 - x) >= epsilon and ans <= x:
          ans += step
          num_guesses += 1
      print("number of guesses = ', num_guesses)
      if abs(ans**2 - x) >= epsilon:
          print("Failed on square root of", x)
      else:
          print(ans, "is close to square root of", x)
    \end{lstlisting}
  \end{block}

\end{frame}

\backmatter

\end{document}
